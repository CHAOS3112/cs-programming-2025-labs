\documentclass{vvsu}

\vvsuyear{2025}

%%%%%%%%%%%%%%%%%%%

\usepackage{graphicx} % для изображений
\usepackage{tabularray} % для таблиц
\usepackage{siunitx} % для обозначений (процент, градус)
\usepackage{listings} % для листингов кода

% Список путей, где будут искаться изображения и файлы
\graphicspath{{images/}}

% Автор документа
\author{А. Е. Филатов}

% Настройка стилей для листингов кода
\input{listing_styles.tex}

%%%%%%%%%%%%%%%%%%%

\begin{document}

% Шапка
\vvsuhead{\linespread{1}\selectfont{}МИНОБРНАУКИ РОССИИ\\
\vspace{10pt}Федеральное государственное бюджетное образовательное учреждение\\
высшего образования\\
\fontsize{13}{13}\selectfont{}<<ВЛАДИВОСТОКСКИЙ ГОСУДАРСТВЕННЫЙ УНИВЕРСИТЕТ>>\\
(ФГБОУ ВО <<ВВГУ>>)\\
\vspace{10pt}\fontsize{12}{12}\selectfont{}ИНСТИТУТ ИНФОРМАЦИОННЫХ ТЕХНОЛОГИЙ И АНАЛИЗА ДАННЫХ\\
КАФЕДРА ИНФОРМАЦИОННЫХ ТЕХНОЛОГИЙ И СИСТЕМ}

% Название отчета
\title{Отчет\\по лабораторной работе №4}
\subtitle{по дисциплине\\<<Информатика и программирование>>}

% Участники работы
\member{Студент\\ гр. БИН-25-3}{А. Е. Филатов}
\member{Ассистент\\ преподавателя}{М.В. Водяницкий}

% Вывод титульника
\maketitle

% Задание
\begin{addition}{Задание}
  Выполнить задания и оформить отчет по стандартам ВВГУ.

  \textit{\textbf{Задание 1.}}  
<<<<<<< HEAD
  Дан список из 10 различных целых чисел. Необходимо найти в нем число 3 и заменить на 30.

  \textit{\textbf{Задание 2.}}  
  Дан список из 5 целых чисел. Необходимо превратить его в список квадратов этих чисел.

  \textit{\textbf{Задание 3.}}  
  Имеется список различных целых чисел. Программа должна найти наибольшее из чисел списка и разделить его на длину списка.
=======
Дан список из 10 различных целых чисел. Необходимо найти в нем число 3 и заменить на 30.

  \textit{\textbf{Задание 2.}}  
 Дан список из 5 целых чисел. Необходимо превратить его в список квадратов этих чисел.

  \textit{\textbf{Задание 3.}}  
Имеется список различных целых чисел. Программа должна найти наибольшее из чисел списка и разделить его на длину списка.
>>>>>>> 62b35d4a3ad8058da0ad97808dfb97566f2dec26

  \textit{\textbf{Задание 4.}}  
  Имеется кортеж из нескольких произвольных элементов. Необходимо этот кортеж отсортировать. Если хотя бы один элемент не является числом, то кортеж остается неизменным.

  \textit{\textbf{Задание 5.}}  
  Имеется словарь товаров в магазине. Необходимо найти товар с минимальной и максимальной ценой.

  \textit{\textbf{Задание 6.}}  
  Имеется список произвольных элементов. Необходимо на основе этого списка создать словарь, где каждый элемент списка будет и ключом, и значением.
  
  \textit{\textbf{Задание 7.}}  
  Имеется словарь перевода английских слов на русский, где ключ английского слово, значение - русского. Необходимо реализовать программу которая получает на ввод русское слово и результатом выдает перевод на английский.


  \textit{\textbf{Задание 8.}}  
  Реализовать игру Камень-Ножницы-Бумага-Ящерица-Спок. Программа должна запрашивать у пользователя ввод одного из вариантов. Второй вариант случайно генерирует сама программа и возвращает победителя.

  Правила игры следующие:
\begin{vvsu_itemize}
    \item Ножницы режут бумагу
    \item Бумага покрывает камень
    \item Камень давит ящерицу
    \item Ящерица отравляет Спока
    \item Спок ломает ножницы
    \item Ножницы обезглавливают ящерицу
    \item Ящерица съедает бумагу
    \item Бумага подставляет Спока
    \item Спок испаряет камень
    \item Камень разбивает ножницы
\end{vvsu_itemize}

  \textit{\textbf{Задание 9.}}  
  Дан список слов - например:
  \begin{vvsu_itemize}
    \item `["яблоко", "груша", "банан", "киви", "апельсин", "ананас"]`
  \end{vvsu_itemize}

  Необходимо создать новый словарь, где:

  \begin{vvsu_itemize}
    \item Ключом будет первая буква слова
    \item Значением - список всех слов, начинающихся с этой буквы
  \end{vvsu_itemize}
  
  Пример результата:\\
    {'я': ['яблоко'], 'г': ['груша'], 'б': ['банан'], 'к': ['киви'], 'а': ['апельсин', 'ананас']}\\
  
  \textit{\textbf{Задание 10.}}  
  Дан список кортежей, где каждый кортеж содержит имя студента и его оценки, например:
  
  // [("Анна", [5, 4, 5]), ("Иван", [3, 4, 4]), ("Мария", [5, 5, 5])]

  Необходимо:

  \begin{vvsu_itemize}
    \item Создать словарь, где ключ - имя студента, значение - средняя оценка
    \item Найти студента с наивысшей средней оценкой
  \end{vvsu_itemize}
  
\end{addition}


% Содержание
\toc

% Глава - Выполнение работы
\section{Выполнение работы}

% Подглава - Задание 1
\subsection{Задание 1}

Сначала обозначим переменную temp, которая принимает значение, введенное пользователем. После чего обрабатываем значение по условиям, если temperature >= 20, то выводим в консоль, кондиционер выключен, в ином случае, выведем, что он включен. На рисунке \ref{fig:code_task_1} представлен код программы.

\begin{vvsu_figure}{Листинг программы для задания 1}{fig:code_task_1}
  \begin{minipage}{.75\textwidth}
    \lstinputlisting[language=Python,basicstyle=\fontsize{10}{10}\linespread{1}\selectfont\ttfamily]{code/5.1.py}
  \end{minipage}
\end{vvsu_figure}

% Подглава - Задание 2
\subsection{Задание 2}

 Запрашиваем ввод номера месяца у пользователя и записываем в переменную num. Далее проверяем корректность переданного номера месяца по условиям: num < 1 и num > 12. При нарушении любого условия выводится сообщение об ошибке. После  с помощью списков с заранее определенными месяцами времен года проверяется принадлежность к времени года с помощью оператора in. На рисунке \ref{fig:code_task_2} представлен код программы.

\begin{vvsu_figure}{Листинг программы для задания 2}{fig:code_task_2}
  \begin{minipage}{.75\textwidth}
    \lstinputlisting[language=Python,basicstyle=\fontsize{10}{10}\linespread{1}\selectfont\ttfamily]{code/5.2.py}
  \end{minipage}
\end{vvsu_figure}

% Подглава - Задание 3
\subsection{Задание 3}

Сначала ввести возраст собаки, записывая значение в переменную dogAge. После проверяем введеные данные на корректность по условию 1 <=dogAge <= 22. Затем конвертируем возраст собаки в человеческий по следующему алгоритму: если dogAge <= 2, то dogAge * 10.5, в ином случае мы получаем следующее выражение - 21 + (dogAge-2)*4. После выводим в консоль результат. На рисунке \ref{fig:code_task_3} представлен код программы.

\begin{vvsu_figure}{Листинг программы для задания 3}{fig:code_task_3}
  \begin{minipage}{.75\textwidth}
    \lstinputlisting[language=Python,basicstyle=\fontsize{10}{10}\linespread{1}\selectfont\ttfamily]{code/5.3.py}
  \end{minipage}
\end{vvsu_figure}

% Подглава - Задание 4
\subsection{Задание 4}

Сперва просим пользователя ввести интересющее его число и запишем его в переменную num. Затем проверим число на четность. После проверяем делится ли сумма цифр числа на 3 без остатка. В случае соблюдения всех условий, выводим в консоль - число делится на 6, в ином случае - не делится. На рисунке \ref{fig:code_task_4} представлен код решения. На рисунке \ref{fig:code_task_4} представлен код решения.

\begin{vvsu_figure}{Листинг программы для задания 4}{fig:code_task_4}
  \begin{minipage}{.75\textwidth}
    \lstinputlisting[language=Python,basicstyle=\fontsize{10}{10}\linespread{1}\selectfont\ttfamily]{code/5.4.py}
  \end{minipage}
\end{vvsu_figure}


% Подглава - Задание 5
\subsection{Задание 5}

Запрашиваем ввод пароля от пользователя и записываем в переменную password. Создаем словарь с условиями проверки пароля: 
Пароль не менее 8 символов; наличие заглавных букв латиницы (any() возвращает True, если хотя бы один символ удовлетворяет условию, c.isupper() проверяет, является ли символ заглавной буквой); наличие строчных букв латиницы; наличие цифр (c.isdigit() проверяет, является ли символ цифрой); проверка наличия специальных символов (not c.isalnum() проверяет, НЕ является ли символ буквой или цифрой). Проверяем каждое условие с помощью цикла (items() возвращает пары (ключ, значение) из словаря).
 Если хотя бы одно условие не выполнено, общий результат становится False – пароль ненадежный. На рисунке \ref{fig:code_task_5} представлен код программы.

\begin{vvsu_figure}{Листинг программы для задания 5}{fig:code_task_5}
  \begin{minipage}{.75\textwidth}
    \lstinputlisting[language=Python,basicstyle=\fontsize{10}{10}\linespread{1}\selectfont\ttfamily]{code/5.5.py}
  \end{minipage}
\end{vvsu_figure}

% Подглава - Задание 6
\subsection{Задание 6}

Год является високосным, если делится на 4 без остатка и не делится на 100 или делится на 400. Проверяем делимость. Оператор and выполняется первым, or выполняется вторым. На рисунке \ref{fig:code_task_6} представлен код программы.

\begin{vvsu_figure}{Листинг программы для задания 6}{fig:code_task_6}
  \begin{minipage}{.75\textwidth}
    \lstinputlisting[language=Python,basicstyle=\fontsize{10}{10}\linespread{1}\selectfont\ttfamily]{code/5.6.py}
  \end{minipage}
\end{vvsu_figure}

% Подглава - Задание 7
\subsection{Задание 7}

Сперва запрашиваем три числа у полльзователя. Далее ищем минимального значения через условные операторы.
Первая проверка:
Проверяет, является ли a меньше или равным b И a меньше или равным c.
Если оба условия истинны, a - наименьшее число.
Вторая проверка:
Выполняется, если первое условие ложно.
Проверяет, является ли b меньше или равным a И b меньше или равным c.
Если истинно, b - наименьшее число.
Третий случай: else:
Срабатывает, если оба предыдущих условия ложны.
Значит, c - наименьшее число
Найденное наименьшее число выводится с использованием f-строки. На рисунке \ref{fig:code_task_7} представлен код программы.

\begin{vvsu_figure}{Листинг программы для задания 7}{fig:code_task_7}
  \begin{minipage}{.75\textwidth}
    \lstinputlisting[language=Python,basicstyle=\fontsize{10}{10}\linespread{1}\selectfont\ttfamily]{code/5.7.py}
  \end{minipage}
\end{vvsu_figure}

% Подглава - Задание 8
\subsection{Задание 8}

Получаем исходную сумму покупки от пользователя для дальнейших расчетов. - преобразует введенную строку в число с плавающей точкой.
Размер скидки зависит от суммы покупки: сумма до 1000р. скидка 0%%; сумма от 1000 до 5000 рублей включительно, скидка: 5%%; сумма от 5001 до 10000 рублей включительно, скидка: 10%%; сумма свыше 10000 рублей, скидка: 15%%. Затем по формуле рассчитывается и выводится итоговая сумма к оплате с учетом скидки. На рисунке \ref{fig:code_task_8} представлен код программы.

\begin{vvsu_figure}{Листинг программы для задания 8}{fig:code_task_8}
  \begin{minipage}{.75\textwidth}
    \lstinputlisting[language=Python,basicstyle=\fontsize{10}{10}\linespread{1}\selectfont\ttfamily]{code/5.8.py}
  \end{minipage}
\end{vvsu_figure}

% Подглава - Задание 9
\subsection{Задание 9}

Создаем функцию time с одним параметром num, параметр num представляет час суток (0-23). После проверяем валидность входных данных:
0 > num < 23; часы не могут быть отрицательными/ в сутках не более 23 часов.
При невалидном вводе выводится сообщение об ошибке.
Если даные коректны, то проверяется принадлежность к времени суток с помощью оператора in и заранее определенных списков с часами каждого из времени суто. На рисунке \ref{fig:code_task_9} представлен код программы.

\begin{vvsu_figure}{Листинг программы для задания 9}{fig:code_task_9}
  \begin{minipage}{.75\textwidth}
    \lstinputlisting[language=Python,basicstyle=\fontsize{10}{10}\linespread{1}\selectfont\ttfamily]{code/5.9.py}
  \end{minipage}
\end{vvsu_figure}

% Подглава - Задание 10
\subsection{Задание 10}

Получаем число от пользователя с помощью input(), int() преобразовывает строку в целое число. При вводе нечисловых данных выводим ошибку.
После того, как мы убедились в корректности данных, начинаем обрабатывать число, а именно: если число <  1 или четное, то оно уже не простое, кроме 2. Далее запускаем цикл, в котором мы будем искать делители нашего числа, если найдутся еще кроме 1 и самого числа, то выводим в консоль, что число составное, в ином случае - простое. На рисунке \ref{fig:code_task_10} представлен код программы.

\begin{vvsu_figure}{Листинг программы для задания 10}{fig:code_task_10}
  \begin{minipage}{.75\textwidth}
    \lstinputlisting[language=Python,basicstyle=\fontsize{10}{10}\linespread{1}\selectfont\ttfamily]{code/5.11.py}
  \end{minipage}
\end{vvsu_figure}

Спасибо за внимание !

\end{document}